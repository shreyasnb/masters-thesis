%%% Macros for definitions, theorems, claims, examples, ...
\usepackage{amsthm}   % key-value interface for enhanced theorem/proof environments
                           % extending `amsthm` (modern implementation of `thmtools` package)

\usepackage{tcolorbox} % flexible frames/boxes
\tcbuselibrary{skins, breakable, theorems}

%% Default style options for `tcolorbox` environments (used in theorems, remarks, ...)
\newlength{\depthofj} \settodepth{\depthofj}{j} % save depth of character "j" (typical descender)
\tcbset{
    skin=enhanced, % for better looking frame breaking
    colframe=black, colback=white,
    boxrule=0.4pt, arc=3pt, boxsep=0em,
    left=0.9em, right=0.9em, top=1.3ex, bottom=1.3ex-0.8\depthofj, % try to reduce effect of descenders
    % enlarge bottom at break by=-\depthofj,
    % pad at break*=3ex,
    beforeafter skip=0.5\baselineskip plus 2pt,
}
\tcbset{ % persist font color from the text before the box, see https://github.com/T-F-S/tcolorbox/issues/305
    every box/.style={coltext=.}, % dot `.` refers to the current text color (just before the box)
}                                 % set inside `every box`, thus using color at the place of box creation

%%% Definitions of theorem/remark-like environments
\newtheoremstyle{thmcommon}% name
  {\topsep}% Space above
  {\topsep}% Space below
  {\normalfont}% Body font
  {}% Indent amount
  {\bfseries\sffamily}% Theorem head font
  {\strut.}% Punctuation after theorem head
  { }% Space after theorem head
  {\thmname{#1}\thmnumber{ #2}\thmnote{ (#3)}}% Theorem head spec

\theoremstyle{thmcommon}

%% More flexible placement of "qed" symbol (star/optional argument tweaks the vertical shift)
%% NOTE: this is used in all theorem/remark-like environments to ensure consistent height
%%       of the final line, particularly when the last line is an equation
\NewCommandCopy{\oldqedhere}{\qedhere}
\RenewDocumentCommand{\qedhere}{s o}{%
    \def\qedshift{0.6ex}%                % by default, set shift corresponfing to normal height display math
    \IfBooleanT{#1}{\def\qedshift{1.5ex}}% if starred, set shift corresponding to big operators (\sum, ...)
    \IfValueT{#2}{\def\qedshift{#2}}%    % if optional argument is passed, set shift to the custom value
    \NewCommandCopy{\oldqed}{\qedsymbol}%
    \renewcommand*{\qedsymbol}{\raisebox{-\qedshift}{\oldqed}}%
    \oldqedhere%
}

%% provide a reference type name for `zref-clever` package
%% NOTE: the plural form is defined only heuristically, tweak manually later if necessary
\NewDocumentCommand{\declareCleverTypeName}{m}{
    \zcRefTypeSetup{#1}{
        name-sg = #1,  Name-sg = \MakeTitlecase{#1},  % singular form (second is capitalized)
        name-pl = #1s, Name-pl = \MakeTitlecase{#1}s, % plural form   (second is capitalized)
    }
}
