%% Ensure consistent height of the final line with `\strut`
%% `\hspace{-1.␣␣em}` to enable longer last line in theorem-like environments,
%%                    otherwise the line is breaking earlier then necesarry
\usepackage{amsthm}
\newcommand*{\theoremqed}{\hspace{-1.195em}\strut}

%% Factory for new theorem-like environments
\NewDocumentCommand{\NewTheoremLike}{O{qed=\theoremqed} m}{
    \declarekeytheorem{#2}[
        sibling=table,   % same numbering as tables/figures
        style=thmcommon, % use the common theorem style
        #1,              % other options, such as `qed` symbol
        tcolorbox-no-titlebar={parbox=false, breakable}, % wrap in a `tcolorbox`, allow breaking
        % TODO: maybe set refname, Refname already here, see https://github.com/mbertucci47/keytheorems/issues/21
    ]
    \declareCleverTypeName{#2} % provide a reference type name
}
\NewDocumentCommand{\NewTheoremLikes}{>{\SplitList{,}}m}{\ProcessList{#1}{\NewTheoremLike}}

\newtheoremstyle{thmcommon}% name
  {\topsep}% Space above
  {\topsep}% Space below
  {\normalfont}% Body font
  {}% Indent amount
  {\bfseries\sffamily}% Theorem head font
  {\strut.}% Punctuation after theorem head
  { }% Space after theorem head
  {\thmname{#1}\thmnumber{ #2}\thmnote{ (#3)}}% Theorem head spec

\theoremstyle{thmcommon}

%% Define theorem-like environments
\NewTheoremLikes{definition, theorem, lemma, corollary, proposition}

%% Tweak plural form of `corollary` reference type name
\zcRefTypeSetup{corollary}{
    name-sg = corollary,   Name-sg = Corollary,   % singular form
    name-pl = corollaries, Name-pl = Corollaries, % plural form
}
